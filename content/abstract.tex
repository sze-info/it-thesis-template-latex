\pagenumbering{roman}
\setcounter{page}{1}

\selecthungarian

%----------------------------------------------------------------------------
% Kivonat Magyarul 
%----------------------------------------------------------------------------
\chapter*{Kivonat}
% TODO: Távolítsd el a megjegyzést, ha mégis szeretnéd, hogy bekerüljön a tartalomjegyzékbe
%\addcontentsline{toc}{chapter}{Kivonat}

Jelen dokumentum a hivatalos BME-VIK diplomaterv és szakdolgozat sablonnak a \textbf{Hibatűrő Rendszerek Kutatócsoport} által karban tartott verzióján alapul. Célja, hogy segítse \sze{} \givk án végző hallgatókat szakdolgozatuk, vagy diplomatervük elkészítésében. Ez a sablon \LaTeX~alapú, a \emph{TeXLive} \TeX-implementációval és a PDF-\LaTeX~fordítóval működőképes. 

Köszönet a Hibatűrő Rendszerek Kutatócsoportnak akik karban tartják a repository-t amin ez a sablon alapul:
\url{https://github.com/FTSRG/thesis-template-latex}


\vfill
\selectenglish


%----------------------------------------------------------------------------
% Abstract in English
%----------------------------------------------------------------------------
\chapter*{Abstract}
% TODO: Távolítsd el a megjegyzést, ha mégis szeretnéd, hogy bekerüljön a tartalomjegyzékbe
%\addcontentsline{toc}{chapter}{Abstract}

This document is a \LaTeX-based skeleton for BSc/MSc~theses based on the official template developed and maintained at the Electrical Engineering and Informatics Faculty, Budapest University of Technology and Economics. The goal of this skeleton is to guide and help students that wish to use \LaTeX{} for their work at \sze{} \givk. It has been tested with the \emph{TeXLive} \TeX~implementation, and it requires the PDF-\LaTeX~compiler.

Many thanks to the Fault Tolerant Systems Research Group who maintain the repository this template is based on: \url{https://github.com/FTSRG/thesis-template-latex}


\vfill
\selectthesislanguage

\newcounter{romanPage}
\setcounter{romanPage}{\value{page}}
\stepcounter{romanPage}