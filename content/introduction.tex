%----------------------------------------------------------------------------
\chapter{\bevezetes}
%----------------------------------------------------------------------------

%----------------------------------------------------------------------------
\section{Diplomakészítési útmutató}
%----------------------------------------------------------------------------

A hallgató a szakdolgozat (B.Sc. szint), illetve a diplomamunka (M.Sc. szint) készítése során végig a jelen Diplomakészítési Útmutató alapján köteles dolgozni.

A dolgozat elkészítésének első lépéseként feltétlenül olvassa el a kari Záróvizsga Szabályzatot (GIVK-ZVSZ) és a Villamosmérnöki szakra vonatkozó egyéb szabályozásokat! A dokumentumok az Automatizálási Tanszék honlapjáról elérhetők (\url{http://automatizalas.sze.hu}).

\subsection{Téma választás}
Az Automatizálási Tanszék mindenkor aktuális szakdolgozat- és diplomamunka-témái a Tanszék honlapján, az Oktatás/Szakdolgozat és diplomaterv menüben találhatók meg. Javasoljuk hallgatóinknak, hogy a honlapot különösen a témaválasztást megelőző időszakban rendszeresen látogassák. A hallgató ezen  témák közül választhat. Amennyiben a hallgató saját témával kíván foglalkozni, úgy azt a felkért konzulensekkel időben köteles egyeztetni. A hallgató által választott témát engedélyeztetni kell a ZVSZ 1/a-b. melléklete szerint (a feladatkiíró lap szakdolgozathoz/diplomamunkához) amely a Tanszék honlapján érhető el.  Az adatlapot a Tanszék titkárságán (C-704) legkésőbb az első diplomakurzus (pl. Szakdolgozat I., Diplomamunka I.) tantárgyfelvételi határidejéig (értsd: péntek 12:00)  köteles leadni, vagy postán eljuttatni.  (A  leadási határidő mindig a vizsgaidőszak első hete péntek 12:00. ) 

Ezzel egy időben a hallgató felveszi a tanszéki konzulens megfelelő kurzusát a Neptun - rendszeren keresztül. A  konzulens felkérése, megkeresése, illetve az adatlap határidőig való leadása a hallgató felelőssége! A feladatkiíró  adatlapot géppel kell kitölteni, majd kézzel aláírni és leadni 2 pld-ban. Nagyon fontos, hogy a határidő után érkező, valamint elektronikus úton küldött formanyomtatványt a Tanszék nem fogad el! A  tanszékvezető és a konzulensek által jóváhagyott és aláírt adatlap egy eredeti példányát a hallgató kap melyet a dolgozatba fűz, egy másolatot a belső konzulens kap, a másik eredetit a tanszéki irattárban kell őrizni.  Fontos elem, hogy a szakdolgozat, illetve a diplomamunka írása során külső konzulenst is fel kell kérni, amiről a hallgatónak időben gondoskodni kell. A félévközi követelményeket a belső konzulens adja meg.

\subsection{Sablon}
A GIVK-en egy \LaTeX{} és egy Word sablon áll rendelkezésre a szakdolgozat, valamint a diplomamunka elkészítésére. Valamelyik sablon használata kötelező, annak minden oldalát aktualizálva a leadandó műbe bekötendő! Figyelmesen olvassa át a sablont, mert egyes részeit aktualizálni kell (pl. név, dolgozat címe, év stb). Az Automatizálási Tanszékre csak e sablon szerint készített dolgozatok adhatók be. A dolgozatot és mellékleteit elektronikus formában is be kell adni CD/DVD-melléklet formájában. Az elkészített dolgozat a kötelező elemeken kívül kb. 60 oldal terjedelmű kell, hogy legyen. Lehetőség van a Tanszék logójának megjelentetésére is, mely letölthető a Tanszék honlapjáról. A szakdolgozat, vagy diplomamunka nyelve lehet a magyartól eltérő is, ha ezt a témaengedélyező lapon is feltüntetik. A dolgozat elkészítése során a hallgató kérjen segítséget konzulensétől, hogy a dolgozat a kritériumoknak minél jobban megfeleljen. 

\subsection{Titkosítás}
Titkosnak minősített dolgozat esetében a titkosítási kérelmet és a titoktartási nyilatkozatot (ZVSZ 3. és 4. melléklet), amennyiben az szükséges, a hallgató készítse elő, és a megfelelő személyekkel írattassa alá. Az Audi Hungáriánál készítendő és titkosítandó szakdolgozatokkal és diplomamunkákkal kapcsolatban külön szabályozás él, amely a tanszéki honlapon megtalálható.

 Az Automatizálási Tanszékre beadandó dolgozat esetén a Titkosítási kérelmet (3. sz. melléklet) a nyilatkozat elé be kell fűzni. 

\subsection{Kivonat}
A mű  rövid, egy oldalas magyar  nyelvű összefoglalását a sablonban megjelölt helyén meg kell adni. Az angol nyelvű összefoglaló is egy oldal terjedelmű. Ezek lényege, egy olyan összefoglaló készítése, amely a dolgozat témáját és annak megvalósítását taglalja. Más egyéb összefoglalót nem kell készíteni!

\subsection{Beadás}
A konzultációs lapot, az értékelő lapot és államvizsga tárgyak bejelentése dokumentumokat a titkárságon (C704)  a dolgozat leadásakor  kell külön leadni.

Az elkészült munkát a jelölt többször olvassa át, ellenőrizze az ábraaláírásokat, az ábrák, az egyenletek, illetve a fejezetek sorszámozását, az irodalmi hivatkozásokat, használja a helyesírás-ellenőrzőt. Ezután adja oda konzulenseinek, akik esetleg javaslatokkal látják el. Ezeket célszerű a végleges dolgozatba beépíteni. Célszerű a dolgozatot laikussal (családtaggal, ismerőssel, baráttal) is elolvastatni, így sok félreérthető rész logikusan felépíthetővé válhat. Ezután következhet a munka bekötése, melyhez szükséges a konzulensek engedélye és aláírása is. A kinyomtatott és beköttetett dolgozatot  egy példányban+ CD melléklet  kell leadni a Tanszék titkárságán., és feltölteni a kijelölt határidőig lib.sze.hu oldalra. A szöveget nyomtathatja egyoldalas, de akár kétoldalas kivitelben, színesben, vagy szürkeárnyalatos formában.

A szakdolgozat, valamint a diplomaterv leadásának határidejét a tanszéki honlapon adjuk meg, helyszíne C-704.  A határidő elmulasztása a soron következő záróvizsgából történő kizárást eredményezi! Az egyes tárgyak Záróvizsga Tematikája a Tanszék honlapján elérhetők az Oktatás/Záróvizsga menüpontban. 

Amennyiben további kérdése merülne fel, kérjük, forduljon konzulenséhez, vagy a tanszékvezetőhöz (ballagi@sze.hu). 

Hallgatóinknak eredményes, szakmai kihívásokban és sikerekben gazdag munkát kívánunk!
