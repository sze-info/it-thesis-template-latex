% !TeX root = ./thesis.tex
% !TeX spellcheck = hu_HU
% !TeX encoding = UTF-8
% !TeX program = pdflatex
%TODO Change language to en_GB (recommended) or en_US for English documents
\documentclass[11pt,a4paper,oneside]{report}             % Egyoldalas (javasolt)
%\documentclass[11pt,a4paper,twoside,openright]{report}  % Duplex

\input{include/packages}


%TODO Saját adataiddal töltsd ki a kommentek szerint
%--------------------------------------------------------------------------------------
\newcommand{\szerzoVezeteknev}{Gipsz}
\newcommand{\szerzoKeresztnev}{Jakab}
\newcommand{\szerzoNeptun}{NEPTUN}

\newcommand{\szakirany}{} % Informatikusoknál nincs szakirány. Villamosmérnököknél: Automatizálás (\aut) vagy Infokommunikáció (\infokom).

\newcommand{\konzulensAMegszolitas}{}
\newcommand{\konzulensAVezeteknev}{Konzulens}
\newcommand{\konzulensAKeresztnev}{Károly}
\newcommand{\konzulensBMegszolitas}{}
\newcommand{\konzulensBVezeteknev}{Konzulens}
\newcommand{\konzulensBKeresztnev}{Kálmán}
\newcommand{\konzulensCMegszolitas}{}
\newcommand{\konzulensCVezeteknev}{}
\newcommand{\konzulensCKeresztnev}{}

\newcommand{\cim}{Dolgozat címe} % Cím
\newcommand{\tanszek}{\szeit} % informatika (\szeit), automatizálási (\szeaut) vagy távközlési (\szetat)
\newcommand{\doktipus}{\szakdolgozat} % Dokumentum típusa (\szakdolgozat, \diplomaterv vagy \dolgozat)
\newcommand{\szak}{\infoBSc} % Mérnökinformatikus BSc (\infoMsc), MSc (\infoMsc), Gazdaságinformatikus BSc (\gazdInfoBsc), MSc (\gazdInfoMsc), vagy Villamosmérnöki BSc (\villBSc), MSc (\villMSc)

%--------------------------------------------------------------------------------------
% Elnevezések
%--------------------------------------------------------------------------------------

\newif\ifen
\newif\ifhu

\newcommand{\en}[1]{\ifen#1\fi}
\newcommand{\hu}[1]{\ifhu#1\fi}


\newcommand{\sze}{%
    \en{Széchenyi István University}%
    \hu{Széchenyi István Egyetem}%
}
\newcommand{\givk}{%
    \en{Faculty of Mechanical Engineering, Informatics and Electrical Engineering}%
    \hu{Gépészmérnöki, Informatikai és Villamosmérnöki Kar}%
}
\newcommand{\szeit}{%
    \en{Department of Informatics}%
    \hu{Informatika Tanszék}%
}
\newcommand{\szeaut}{%
    \en{Department of Automation}%
    \hu{Automatizálási Tanszék}%
}
\newcommand{\szetat}{%
    \en{Department of Telecommunications}%
    \hu{Távközlési Tanszék}%
}
\newcommand{\aut}{%
    \en{Specialization in Automation}%
    \hu{Automatizálási Szakirány}%
}
\newcommand{\infokom}{%
    \en{Specialization in Infocommunication}%
    \hu{Infokommunikáció Szakirány}%
}
\newcommand{\keszitette}{%
    \en{Author}%
    \hu{Készítette}%
}
\newcommand{\konzulens}{%
    \en{Advisor}%
    \hu{Konzulens}%
}
\newcommand{\szakdolgozat}{%
    \en{Bachelor's Thesis}%
    \hu{Szakdolgozat}%
}
\newcommand{\diplomaterv}{%
    \en{Master's Thesis}%
    \hu{Diplomamunka}%
}
\newcommand{\dolgozat}{%
    \en{Project}%
    \hu{Dolgozat}%
}
\newcommand{\infoBSc}{%
    \en{Computer Science Engineering, BSc}%
    \hu{Mérnökinformatikus BSc}%
}
\newcommand{\infoMSc}{%
    \en{Computer Science Engineering, MSc}%
    \hu{Mérnökinformatikus MSc}%
}
\newcommand{\gazdInfoBSc}{%
    \en{Business Informatics, BSc}%
    \hu{Gazdaságinformatikus BSc}%
}
\newcommand{\gazdInfoMSc}{%
    \en{Business Informatics, MSc}%
    \hu{Gazdaságinformatikus MSc}%
}
\newcommand{\villBSc}{%
    \en{Electrical Engineering BSc}%
    \hu{Villamosmérnöki BSc}%
}
\newcommand{\villMSc}{%
    \en{Electrical Engineering MSc}%
    \hu{Villamosmérnöki MSc}%
}
\newcommand{\pelda}{%
    \en{Example}%
    \hu{Példa}%
}
\newcommand{\definicio}{%
    \en{Definition}%
    \hu{Definíció}%
}
\newcommand{\tetel}{%
    \en{Theorem}%
    \hu{Tétel}%
}
\newcommand{\bevezetes}{%
    \en{Introduction}%
    \hu{Bevezetés}%
}
\newcommand{\koszonetnyilvanitas}{%
    \en{Acknowledgements}%
    \hu{Köszönetnyilvánítás}%
}
\newcommand{\fuggelek}{%
    \en{Appendix}%
    \hu{Függelék}%
}

\newcommand{\szerzo}{%
    \en{\szerzoKeresztnev{} \szerzoVezeteknev}%
    \hu{\szerzoVezeteknev{} \szerzoKeresztnev}%
}
\newcommand{\konzulensA}{%
    \en{\konzulensAMegszolitas\konzulensAKeresztnev{} \konzulensAVezeteknev}%
    \hu{\konzulensAMegszolitas\konzulensAVezeteknev{} \konzulensAKeresztnev}%
}
\newcommand{\konzulensB}{%
    \en{\konzulensBMegszolitas\konzulensBKeresztnev{} \konzulensBVezeteknev}%
    \hu{\konzulensBMegszolitas\konzulensBVezeteknev{} \konzulensBKeresztnev}%
}
\newcommand{\konzulensC}{%
    \en{\konzulensCMegszolitas\konzulensCKeresztnev{} \konzulensCVezeteknev}%
    \hu{\konzulensCMegszolitas\konzulensCVezeteknev{} \konzulensCKeresztnev}%
}

%TODO Nyelv beállítása
% Beállítások magyar nyelvű dolgozathoz

\hutrue

\newcommand{\selectthesislanguage}{
    \selecthungarian
    \enfalse
    \hutrue
}

\bibliographystyle{huplain}

% Opcionálisan átnevezhető címek
%\addto\captionsmagyar{%
%\renewcommand{\listfigurename}{Saját ábrajegyzék cím}
%\renewcommand{\listtablename}{Saját táblázatjegyzék cím}
%\renewcommand{\bibname}{Saját irodalomjegyzék név}
%}

\def\lstlistingname{lista}

\newcommand{\appendixnumber}{6}  % a fofejezet-szamlalo az angol ABC 6. betuje (F) lesz

% Settings for English documents
%
\entrue

\newcommand{\selectthesislanguage}{
    \selectenglish
    \hufalse
    \entrue
}

\bibliographystyle{plainnat}

% Optional custom titles
%\addto\captionsenglish{%
%\renewcommand*{\listfigurename}{Your list of figures title}
%\renewcommand*{\listtablename}{Your list of tables title}
%\renewcommand*{\bibname}{Your bibliography title}
%}

\newcommand{\ie}{i.e.\@\xspace}
\newcommand{\Ie}{I.e.\@\xspace}
\newcommand{\eg}{e.g.\@\xspace}
\newcommand{\Eg}{E.g.\@\xspace}
\newcommand{\etal}{et al.\@\xspace}
\newcommand{\etc}{etc.\@\xspace}
\newcommand{\vs}{vs.\@\xspace}
\newcommand{\viz}{viz.\@\xspace} % videlicet
\newcommand{\cf}{cf.\@\xspace} % confer
\newcommand{\Cf}{Cf.\@\xspace}
\newcommand{\wrt}{w.r.t.\@\xspace} % with respect to

\newcommand{\appendixnumber}{1}  % a fofejezet-szamlalo az angol ABC 1. betuje (A) lesz



\newcommand{\szerzoMeta}{\szerzoVezeteknev{} \szerzoKeresztnev} % egy szerző esetén TODO@FMA két szerző
\input{include/preamble} % beállítások, nem kell vele foglalkoznod remélhetőleg, de ha valami latex hekkelésre vagy új parancsra van szükséged annak itt a helye


%--------------------------------------------------------------------------------------
% Itt kezdődik a dolgozat
%--------------------------------------------------------------------------------------
\begin{document}

\selectthesislanguage

% Külső borító, minta kötéshez - csak elektronikus leadás esetén eltávolítandó
%~~~~~~~~~~~~~~~~~~~~~~~~~~~~~~~~~~~~~~~~~~~~~~~~~~~~~~~~~~~~~~~~~~~~~~~~~~~~~~~~~~~~~~
\hypersetup{pageanchor=false}
%--------------------------------------------------------------------------------------
%	The outer cover page
%--------------------------------------------------------------------------------------
\pagenumbering{gobble}
\begin{center}

\vspace{130pt} %because it's the top
{\Large \bfseries \sze \\ \givk \\ \tanszek }\\
\vspace{180pt}
{\huge \bfseries \MakeUppercase {\doktipus}}\\
\vspace{160pt}
{\huge \bfseries{\szerzo}}

\vspace{40pt}
\Large \textbf{\szak{}}\\
\textbf{\szakirany}\\
\vfill
{\Large \textbf{\the\year}}

\end{center}
\hypersetup{pageanchor=false}

 

% Címoldal 
%~~~~~~~~~~~~~~~~~~~~~~~~~~~~~~~~~~~~~~~~~~~~~~~~~~~~~~~~~~~~~~~~~~~~~~~~~~~~~~~~~~~~~~
\hypersetup{pageanchor=false}
%--------------------------------------------------------------------------------------
%	The title page
%--------------------------------------------------------------------------------------
\begin{titlepage}

\ifthenelse{\equal{\tanszek}{\szeit}}{
    \includegraphics[height=15mm,keepaspectratio]{figures/sze_logo_wide.pdf} 
    \hfill
    \includegraphics[height=15mm,keepaspectratio]{figures/dept_it_logo.png}
}{
    \includegraphics[width=110mm,keepaspectratio]{figures/sze_logo_wide.pdf}
}

\begin{center}

\vspace{130pt} %because it's the top
{\Huge \bfseries \MakeUppercase {\doktipus}}\\
\vspace{80pt}
{\huge \bfseries \cim}\\
\vspace{80pt}
{\huge \bfseries{\szerzo}}

\vspace{80pt}
\Large \textbf{\szak{}}\\
\textbf{\szakirany}\\
\vfill
{\Large \textbf{\the\year}}

\end{center}
\end{titlepage}
\hypersetup{pageanchor=false}



%TODO Feladatkiíró lap helye, csak a nyomtatott verzijóba kerül az eredeti példány
%~~~~~~~~~~~~~~~~~~~~~~~~~~~~~~~~~~~~~~~~~~~~~~~~~~~~~~~~~~~~~~~~~~~~~~~~~~~~~~~~~~~~~~
\pagenumbering{gobble}
%--------------------------------------------------------------------------------------
% Feladatkiiras (a tanszeken atveheto, kinyomtatott valtozat)
%--------------------------------------------------------------------------------------
\clearpage

%--------------------------------------------------------------------------------------
% TODO: végleges változatból töröld ezt a részt
\begin{center}
\large
\textbf{FELADATKIÍRÁS}\\
\end{center}
A feladatkiíró lapot két példányban kell leadni a tanszéki adminisztrációban. Beadás előtt az egyiket visszakapod és a leadott munkába eredeti, tanszéki pecséttel ellátott és a tanszékvezető által aláírt lapot kell belefűzni (ezen oldal \emph{helyett}, ez az oldal csak útmutatás).
%--------------------------------------------------------------------------------------

% PDF formátumú leírás esetén
%\includepdf{figures/kiiras.pdf}

% Képfájlokhoz
% \includegraphics*[width=\linewidth]{figures/kiiras.png}


% Nyilatkozat és Kivonat
%~~~~~~~~~~~~~~~~~~~~~~~~~~~~~~~~~~~~~~~~~~~~~~~~~~~~~~~~~~~~~~~~~~~~~~~~~~~~~~~~~~~~~~
\include{include/declaration} % ez legenerálódik magától a fentebb megadott adatok alapján
\pagenumbering{roman}
\setcounter{page}{1}

\selecthungarian

%----------------------------------------------------------------------------
% Kivonat Magyarul 
%----------------------------------------------------------------------------
\chapter*{Kivonat}
% TODO: Távolítsd el a megjegyzést, ha mégis szeretnéd, hogy bekerüljön a tartalomjegyzékbe
%\addcontentsline{toc}{chapter}{Kivonat}

Jelen dokumentum a hivatalos BME-VIK diplomaterv és szakdolgozat sablonnak a \textbf{Hibatűrő Rendszerek Kutatócsoport} által karban tartott verzióján alapul. Célja, hogy segítse \sze{} \givk án végző hallgatókat szakdolgozatuk, vagy diplomatervük elkészítésében. Ez a sablon \LaTeX~alapú, a \emph{TeXLive} \TeX-implementációval és a PDF-\LaTeX~fordítóval működőképes. 

Köszönet a Hibatűrő Rendszerek Kutatócsoportnak akik karban tartják a repository-t amin ez a sablon alapul:
\url{https://github.com/FTSRG/thesis-template-latex}


\vfill
\selectenglish


%----------------------------------------------------------------------------
% Abstract in English
%----------------------------------------------------------------------------
\chapter*{Abstract}
% TODO: Távolítsd el a megjegyzést, ha mégis szeretnéd, hogy bekerüljön a tartalomjegyzékbe
%\addcontentsline{toc}{chapter}{Abstract}

This document is a \LaTeX-based skeleton for BSc/MSc~theses based on the official template developed and maintained at the Electrical Engineering and Informatics Faculty, Budapest University of Technology and Economics. The goal of this skeleton is to guide and help students that wish to use \LaTeX{} for their work at \sze{} \givk. It has been tested with the \emph{TeXLive} \TeX~implementation, and it requires the PDF-\LaTeX~compiler.

Many thanks to the Fault Tolerant Systems Research Group who maintain the repository this template is based on: \url{https://github.com/FTSRG/thesis-template-latex}


\vfill
\selectthesislanguage

\newcounter{romanPage}
\setcounter{romanPage}{\value{page}}
\stepcounter{romanPage} %TODO ezt át kell írnod

% Tartalomjegyzék
%~~~~~~~~~~~~~~~~~~~~~~~~~~~~~~~~~~~~~~~~~~~~~~~~~~~~~~~~~~~~~~~~~~~~~~~~~~~~~~~~~~~~~~
\tableofcontents\vfill

% A dolgozat lényegi része
%~~~~~~~~~~~~~~~~~~~~~~~~~~~~~~~~~~~~~~~~~~~~~~~~~~~~~~~~~~~~~~~~~~~~~~~~~~~~~~~~~~~~~~
\pagenumbering{arabic}

%TODO készítsd el a saját munkád
%----------------------------------------------------------------------------
\chapter{\bevezetes}
%----------------------------------------------------------------------------

%----------------------------------------------------------------------------
\section{Diplomakészítési útmutató}
%----------------------------------------------------------------------------

A hallgató a szakdolgozat (B.Sc. szint), illetve a diplomamunka (M.Sc. szint) készítése során végig a jelen Diplomakészítési Útmutató alapján köteles dolgozni.

A dolgozat elkészítésének első lépéseként feltétlenül olvassa el a kari Záróvizsga Szabályzatot (GIVK-ZVSZ) és a Villamosmérnöki szakra vonatkozó egyéb szabályozásokat! A dokumentumok az Automatizálási Tanszék honlapjáról elérhetők (\url{http://automatizalas.sze.hu}).

\subsection{Téma választás}
Az Automatizálási Tanszék mindenkor aktuális szakdolgozat- és diplomamunka-témái a Tanszék honlapján, az Oktatás/Szakdolgozat és diplomaterv menüben találhatók meg. Javasoljuk hallgatóinknak, hogy a honlapot különösen a témaválasztást megelőző időszakban rendszeresen látogassák. A hallgató ezen  témák közül választhat. Amennyiben a hallgató saját témával kíván foglalkozni, úgy azt a felkért konzulensekkel időben köteles egyeztetni. A hallgató által választott témát engedélyeztetni kell a ZVSZ 1/a-b. melléklete szerint (a feladatkiíró lap szakdolgozathoz/diplomamunkához) amely a Tanszék honlapján érhető el.  Az adatlapot a Tanszék titkárságán (C-704) legkésőbb az első diplomakurzus (pl. Szakdolgozat I., Diplomamunka I.) tantárgyfelvételi határidejéig (értsd: péntek 12:00)  köteles leadni, vagy postán eljuttatni.  (A  leadási határidő mindig a vizsgaidőszak első hete péntek 12:00. ) 

Ezzel egy időben a hallgató felveszi a tanszéki konzulens megfelelő kurzusát a Neptun - rendszeren keresztül. A  konzulens felkérése, megkeresése, illetve az adatlap határidőig való leadása a hallgató felelőssége! A feladatkiíró  adatlapot géppel kell kitölteni, majd kézzel aláírni és leadni 2 pld-ban. Nagyon fontos, hogy a határidő után érkező, valamint elektronikus úton küldött formanyomtatványt a Tanszék nem fogad el! A  tanszékvezető és a konzulensek által jóváhagyott és aláírt adatlap egy eredeti példányát a hallgató kap melyet a dolgozatba fűz, egy másolatot a belső konzulens kap, a másik eredetit a tanszéki irattárban kell őrizni.  Fontos elem, hogy a szakdolgozat, illetve a diplomamunka írása során külső konzulenst is fel kell kérni, amiről a hallgatónak időben gondoskodni kell. A félévközi követelményeket a belső konzulens adja meg.

\subsection{Sablon}
A GIVK-en egy \LaTeX{} és egy Word sablon áll rendelkezésre a szakdolgozat, valamint a diplomamunka elkészítésére. Valamelyik sablon használata kötelező, annak minden oldalát aktualizálva a leadandó műbe bekötendő! Figyelmesen olvassa át a sablont, mert egyes részeit aktualizálni kell (pl. név, dolgozat címe, év stb). Az Automatizálási Tanszékre csak e sablon szerint készített dolgozatok adhatók be. A dolgozatot és mellékleteit elektronikus formában is be kell adni CD/DVD-melléklet formájában. Az elkészített dolgozat a kötelező elemeken kívül kb. 60 oldal terjedelmű kell, hogy legyen. Lehetőség van a Tanszék logójának megjelentetésére is, mely letölthető a Tanszék honlapjáról. A szakdolgozat, vagy diplomamunka nyelve lehet a magyartól eltérő is, ha ezt a témaengedélyező lapon is feltüntetik. A dolgozat elkészítése során a hallgató kérjen segítséget konzulensétől, hogy a dolgozat a kritériumoknak minél jobban megfeleljen. 

\subsection{Titkosítás}
Titkosnak minősített dolgozat esetében a titkosítási kérelmet és a titoktartási nyilatkozatot (ZVSZ 3. és 4. melléklet), amennyiben az szükséges, a hallgató készítse elő, és a megfelelő személyekkel írattassa alá. Az Audi Hungáriánál készítendő és titkosítandó szakdolgozatokkal és diplomamunkákkal kapcsolatban külön szabályozás él, amely a tanszéki honlapon megtalálható.

 Az Automatizálási Tanszékre beadandó dolgozat esetén a Titkosítási kérelmet (3. sz. melléklet) a nyilatkozat elé be kell fűzni. 

\subsection{Kivonat}
A mű  rövid, egy oldalas magyar  nyelvű összefoglalását a sablonban megjelölt helyén meg kell adni. Az angol nyelvű összefoglaló is egy oldal terjedelmű. Ezek lényege, egy olyan összefoglaló készítése, amely a dolgozat témáját és annak megvalósítását taglalja. Más egyéb összefoglalót nem kell készíteni!

\subsection{Beadás}
A konzultációs lapot, az értékelő lapot és államvizsga tárgyak bejelentése dokumentumokat a titkárságon (C704)  a dolgozat leadásakor  kell külön leadni.

Az elkészült munkát a jelölt többször olvassa át, ellenőrizze az ábraaláírásokat, az ábrák, az egyenletek, illetve a fejezetek sorszámozását, az irodalmi hivatkozásokat, használja a helyesírás-ellenőrzőt. Ezután adja oda konzulenseinek, akik esetleg javaslatokkal látják el. Ezeket célszerű a végleges dolgozatba beépíteni. Célszerű a dolgozatot laikussal (családtaggal, ismerőssel, baráttal) is elolvastatni, így sok félreérthető rész logikusan felépíthetővé válhat. Ezután következhet a munka bekötése, melyhez szükséges a konzulensek engedélye és aláírása is. A kinyomtatott és beköttetett dolgozatot  egy példányban+ CD melléklet  kell leadni a Tanszék titkárságán., és feltölteni a kijelölt határidőig lib.sze.hu oldalra. A szöveget nyomtathatja egyoldalas, de akár kétoldalas kivitelben, színesben, vagy szürkeárnyalatos formában.

A szakdolgozat, valamint a diplomaterv leadásának határidejét a tanszéki honlapon adjuk meg, helyszíne C-704.  A határidő elmulasztása a soron következő záróvizsgából történő kizárást eredményezi! Az egyes tárgyak Záróvizsga Tematikája a Tanszék honlapján elérhetők az Oktatás/Záróvizsga menüpontban. 

Amennyiben további kérdése merülne fel, kérjük, forduljon konzulenséhez, vagy a tanszékvezetőhöz (ballagi@sze.hu). 

Hallgatóinknak eredményes, szakmai kihívásokban és sikerekben gazdag munkát kívánunk!

%----------------------------------------------------------------------------
\chapter{A dolgozatról}
%----------------------------------------------------------------------------

%----------------------------------------------------------------------------
\section{A dolgozat célja}
%----------------------------------------------------------------------------
A szakdolgozat és a diplomamunka célja annak bizonyítása, hogy a jelölt önálló mérnöki munkára képes. Az elkészített mű tehát saját alkotómunkát kell, hogy bizonyítson!

%----------------------------------------------------------------------------
\section{A dolgozat felépítése}
%----------------------------------------------------------------------------
A dolgozat sorszámozott fejezetekből, illetve alfejezetekből áll. A tartalomjegyzék a sablon szerint a dolgozat elején legyen. A dolgozat néhány oldalas bevezetővel kezdődjék, amely bemutatja a feldolgozott szakterületet, ha szükséges történelmi utalásokat is tehet, s a jelölt itt jusson el a megoldandó probléma világos, tényszerű megfogalmazásához, s vázolja fel, hogy azt milyen módon oldotta meg. Ha szükséges, röviden bemutathatja a céget, ahol a munkát végezte, méltathatja a felvetett probléma időszerűségét, a megoldás korszerűségét. A bevezető célja, hogy a bíráló, vagy az olvasó el tudja helyezni az elkészített munkát a szakmán belül. A dolgozat fejezeteinek számozását a Bevezető fejezettel kezdje.

A felhasznált szakirodalomnak a jelölt által történő feldolgozása és bemutatása rendkívül fontos, hiszen a munkát arra alapozva készíti el. Szükséges tehát egy olyan fejezet megírása is (amely rendre a bevezetést követi), amelyben a jelölt a szakirodalomra (szakkönyvek, szakcikkek, tankönyvek) hivatkozva összegzi a már ismert tényeket, eredményeket és összefüggéseket. A felhasznált irodalom feldolgozásáról szóló fejezet ne a jól ismert tananyag ismétlése legyen! Törekedjen arra, hogy a fejezet áttekintése után az olvasó elegendő ismerettel rendelkezzen ahhoz, hogy a jelölt saját munkáját megértse. Használja a könyvtárat, s válogasson az interneten közzétett anyagok között, de kerülje a kétes eredetű forrásokat! Ez a fejezet kb. 10-15 oldal terjedelmű legyen.

A következő fejezet az elvégzett munkát hivatott bemutatni, s így a terjedelme is nagyobb kell, hogy legyen. A dolgozat írásakor ezen fejezetben nyugodtan használhat egyes szám első személyt (pl. megoldottam, megterveztem stb.), hiszen a munka a sajátja. A fejezetet célszerűen a feladat részletes leírásával kezdje, térjen ki minden lényeges momentumra. Gyakran előfordul, hogy a feladat egy meglevő rendszer átalakítása, bővítése. Ilyenkor a meglevő rendszer ismertetése a fejezet elején történjen meg, a változtatások bemutatása, a tervezés menete, az elvégzett lépések indoklása stb. pedig a fejezet fő súlypontját alkossák. Ebben a fejezetben fotókon, ábrákon, grafikonokon, képleteken keresztül érthetően, világosan mutassa be, hogy mi a saját, önálló tevékenysége, mit és hogyan oldott meg, azokból milyen eredmények születtek. A fejezet végén elemezze az elkészült munkát. Ez a fejezet legyen kb. 30-40 oldal, s ez legyen a dolgozat hangsúlyos része. Amennyiben munkája több, határozottan elkülönülő tevékenységből állt, ez a rész több fejezetre is tagolható.

A dolgozatot az összefoglalás zárja. Itt múlt időben a szerző röviden ismételje meg, hogy mit és hogy valósított meg. Ebben a rövid, egy-két oldalas fejezetben a jelölt rámutathat a még megoldásra váró kérdésekre, esetleges jövőbeni tervekre, feladatokra. Írja le tapasztalatait, következtetéseit.

A következő szakasz az irodalomjegyzék, amelynek formája kötött. A Tanszék megkötése, hogy az internetes források száma nem érheti el a teljes irodalmi hivatkozások számának 30\%-át, továbbá internetes forrás megjelölésekor kötelező a honlap utolsó látogatásának időpontját is megadni! Az irodalmi hivatkozásokat a szövegben a megfelelő helyen jelölni kell. A szerzők nevét mindenütt “Családnév, X.” formában kell megadni, ahol X. a szerző keresztnevének (keresztneveinek) kezdőbetűje. Magyar cikk esetén a vessző a családnév és a keresztnév kezdőbetűje közt elhagyható. Ha az egyértelműség megkívánja, a keresztnév kiírható teljesen is. Az irodalmi hivatkozások \ref{sec:HowtoReference} fejezetben bővebben kitérünk.

%----------------------------------------------------------------------------
\section{Formai követelmények}
%----------------------------------------------------------------------------
A \LaTeX{} sablon előnye, hogy ezzel nem kell foglalkoznod. Ha rendeltetésszerűen használod a sablont, akkor formai szempontból a dolgozat megfelelő lesz.

%TODO@FMA: SZE követelmenyek
%----------------------------------------------------------------------------
\section{A dolgozat nyelve}
%----------------------------------------------------------------------------
Mivel Magyarországon a hivatalos nyelv a magyar, ezért alapértelmezésben magyarul kell megírni a dolgozatot. Aki külföldi posztgraduális képzésben akar részt venni, nemzetközi szintű tudományos kutatást szeretne végezni, vagy multinacionális cégnél akar elhelyezkedni, annak célszerű angolul megírnia diplomadolgozatát. Mielőtt a hallgató az angol nyelvű verzió mellett dönt, erősen ajánlott mérlegelni, hogy ez mennyi többletmunkát fog a hallgatónak jelenteni fogalmazás és nyelvhelyesség terén, valamint -- nem utolsó sorban -- hogy ez mennyi többletmunkát fog jelenteni a konzulens illetve bíráló számára. Egy nehezen olvasható, netalán érthetetlen szöveg teher minden játékos számára.

%TODO@FMA: SZE követelmenyek
%----------------------------------------------------------------------------
\section{A dokumentum nyomdatechnikai kivitele}
%----------------------------------------------------------------------------
A dolgozatot A4-es fehér lapra nyomtatva, 2,5 centiméteres margóval (+1~cm kötésbeni), 11--12 pontos betűmérettel, talpas betűtípussal és másfeles sorközzel célszerű elkészíteni.

Annak érdekében, hogy a dolgozat külsőleg is igényes munka benyomását keltse, érdemes figyelni az alapvető tipográfiai szabályok betartására~\cite{Jeney}.

\include{content/latex-tools}
\include{content/template-usage}


% Köszönetnyilvánítás - opcionális
%~~~~~~~~~~~~~~~~~~~~~~~~~~~~~~~~~~~~~~~~~~~~~~~~~~~~~~~~~~~~~~~~~~~~~~~~~~~~~~~~~~~~~~
\include{content/acknowledgement}


% Ábrák listája - a word-ös sablon szerint szükséges
%~~~~~~~~~~~~~~~~~~~~~~~~~~~~~~~~~~~~~~~~~~~~~~~~~~~~~~~~~~~~~~~~~~~~~~~~~~~~~~~~~~~~~~
\listoffigures\addcontentsline{toc}{chapter}{\listfigurename}


% Táblázatok listája - opcionális
%~~~~~~~~~~~~~~~~~~~~~~~~~~~~~~~~~~~~~~~~~~~~~~~~~~~~~~~~~~~~~~~~~~~~~~~~~~~~~~~~~~~~~~
%\listoftables\addcontentsline{toc}{chapter}{\listtablename}


% Irodalomjegyzék
%~~~~~~~~~~~~~~~~~~~~~~~~~~~~~~~~~~~~~~~~~~~~~~~~~~~~~~~~~~~~~~~~~~~~~~~~~~~~~~~~~~~~~~
\addcontentsline{toc}{chapter}{\bibname}
\bibliography{bib/mybib}


% Függelékek
%~~~~~~~~~~~~~~~~~~~~~~~~~~~~~~~~~~~~~~~~~~~~~~~~~~~~~~~~~~~~~~~~~~~~~~~~~~~~~~~~~~~~~~
\include{content/appendices}

%\label{page:last}
\end{document}
